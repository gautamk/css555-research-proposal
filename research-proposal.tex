\documentclass[dvips,12pt]{article}

% Any percent sign marks a comment to the end of the line

% Every latex document starts with a documentclass declaration like this
% The option dvips allows for graphics, 12pt is the font size, and article
%   is the style

\usepackage[pdftex]{graphicx}
\usepackage{url}
% These are additional packages for "pdflatex", graphics, and to include
% hyperlinks inside a document.

\setlength{\oddsidemargin}{0.25in}
\setlength{\textwidth}{6.5in}
\setlength{\topmargin}{0in}
\setlength{\textheight}{8.5in}
\renewcommand{\refname}{Annotated Biblography}
% These force using more of the margins that is the default style

\begin{document}

% Everything after this becomes content
% Replace the text between curly brackets with your own

\title{Research Proposal: Digging deeper into Software maintainability}
\author{Gautam Kumar}
\date{\today}

% You can leave out "date" and it will be added automatically for today
% You can change the "\today" date to any text you like

\maketitle

% This command causes the title to be created in the document

\section{Introduction}

Software maintenance can be defined as any modification to the product after delivery to the customer. Maintenance is considered the most expensive portion of a project's life-cycle and is known to take up around 40\% to 80\% of the total costs of the project according to a paper by Robert Glass \cite{glass_frequently_2001}. 

A 1999 study\cite{zhang_analysis_2000} after analysing 32 factors found that Software complexity, Programmer skill, testability and test coverage to the major factors affecting Software reliability and maintainability.

Considering the critical role played by maintainance in a software project's lifecycle I would like to focus my research on uncovering the influence of software models and quality attributes on Maintainability.

\section{Description}

My research into maintainability would consist of two sections. The first would be an analysis of software models and their influence on maintainability. For example in the paper "Long-term Life Cycle Impact of Agile Methodologies" \cite{kajko-mattsson_long-term_2006} Grace Lewis mentions that maintainability as a quality attiribute has to be baked into the system because of the small role that software architecture plays in agile methodologies. 

The second section would be research into the effects of other quality attributes such as Useability, testability and Reuseability on Maintainability. For example in a paper discussing the effects dependency injection \cite{razina_effects_2007} the authors note how decreased coupling of software components improves testability and also maintainability. 

\section{Relevance}

Maintainability is a core quality attribute in many software models and has far reaching consequences into the success of any project. I feel that this research aligns well with some of the learning objectives of the class such as ``Research best practices and methods to educate a technical audience.'' and ``Create \& evaluate models and other artifacts used to express software design ''.
I hope to better understand the value and effects using various software quality models at the end of this research work. 

\nocite{dehaghani_which_2013}

\bibliographystyle{annotate}
\bibliography{research-proposal}


\end{document}