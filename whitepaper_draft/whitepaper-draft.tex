\documentclass[dvips,12pt]{article}

% Any percent sign marks a comment to the end of the line

% Every latex document starts with a documentclass declaration like this
% The option dvips allows for graphics, 12pt is the font size, and article
%   is the style

\usepackage[pdftex]{graphicx}
\usepackage{url}
\usepackage{dirtytalk}
% These are additional packages for "pdflatex", graphics, and to include
% hyperlinks inside a document.

\setlength{\oddsidemargin}{0.25in}
\setlength{\textwidth}{6.5in}
\setlength{\topmargin}{0in}
\setlength{\textheight}{8.5in}
\renewcommand{\refname}{Biblography}
% These force using more of the margins that is the default style

\begin{document}

% Everything after this becomes content
% Replace the text between curly brackets with your own

\title{Whitepaper Draft\\
TODO: Title}
\author{Gautam Kumar\\
Instructor: Prof. Mark Kochanski}
\date{\today}

% You can leave out "date" and it will be added automatically for today
% You can change the "\today" date to any text you like

\maketitle
\newpage
\tableofcontents
\newpage
% This command causes the title to be created in the document

\section{Software Maintenance and Maintainability}
Software maintenance as defined by the International Organization for Standardization is the process of modifying a software product after delivery to correct faults or to improve any quality attributes \cite{_international_2006} \\

Maintainability is a quality attribute that describes the ease of introducing modifications into the software systems. Maintainability has four sub-characteristics namely analysability, changeability, stability and testability\cite{_international_2006}.\\

\subsection{Types of Maintenance}
ISO defines four types of software maintenance, Corrective, Preventive, Adaptive and Perfective. \\

Corrective maintenance refers to modifications done fix actual errors in the product while preventive maintenance is performed to introduce changes that can potentially prevent problems during the normal operation of the software product. \\

Adaptive and perfective maintenance are modifications introduced to enhance the core functionality of the software by improving one or more of its quality attributes. 

\section{Quantifying Maintainability}
Understanding maintainability as a quality attribute is useful in learning the benefits of  its presence in a product but to derive value from the process of evaluating software maintainability we require a method of measuring maintainability as a metric.

\subsection{UML based metrics}
One way to measure maintainability is to analyse the class diagram. This is especially useful in projects which use an object oriented programming language and have an existing UML diagram describing the software structure. \\


Genero et al \cite{genero_building_2003} describe an elegant way of predicting the maintainability of software early in the design phase by measuring the size and structural complexity of the software project. \\

UML based metrics cannot independently describe the maintainability of a system because they suffer from the same basic flaw that UML adoption faces. Its the fact that UML isn't as widely adopted in an industry setting as one would expect. Though UML is extremely well defined and documented its complexity and the learning curve needed to effectively adopt UML in an organisation setting serves as a serious impediment to its adoption which in turn reduces the value of using an UML based metric in measuring the maintainability of a software architecture.


\subsection{Related quality attributes}
Maintainability has a positive relationship with availability, flexibility, reliability, testability and a negative relationship with efficiency \cite{karl_software_2003}. This means that an increase in flexibility, for example, could potentially improve the maintainability of the software system. An increase in efficiency could potentially hamper maintainability.\\

Understanding these trade off is important because indirect measurements as a factor of one of the aforementioned quality attributes can play a critical role when direct measurements of maintainability aren't available.


%\section{Whitepaper Outline}
%\begin{enumerate}
%	\item Introduction
%	\begin{enumerate}
%		\item Problem Statement
%		\item Relevance and reasoning
%		\item Definition of uncommon / ambigious terms
%	\end{enumerate}
%	\item Software Maintainance
%		\begin{enumerate}
%			\item Describe Software maintainance and maintainability.
%			\item Define attributes which affect software maintainance.
%			\item Define metrics to quantify maintainability and its related attributes
%		\end{enumerate}
%	\item Analysis of Software Models
%	\begin{enumerate}
%		\item Describe a few common Software Models
%		\item Apply metrics calculations on Software Models
%		\item Analyse and interpret results. 
%	\end{enumerate}
%	\item Analysis of Software Architectures
%	\begin{enumerate}
%		\item Describe a few common Software Architectures
%		\item Compute maintainability metrics of common software architectures
%		\item Analyse and interpret results.
%	\end{enumerate}
%	\item Conclusion
%	\begin{enumerate}
%		\item Conclusions
%		\item Recommendations if any
%	\end{enumerate}
%\end{enumerate}

\newpage
\bibliographystyle{acm}
\bibliography{whitepaper-draft}


\end{document}